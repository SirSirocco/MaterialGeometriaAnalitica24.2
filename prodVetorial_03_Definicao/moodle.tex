Talvez pareça estranho, mas me permitam escrever

\( \left( \left \{ \begin{array}{rcl} \vec{v} \cdot \vec{w} &=& 0 \\ \vec{u} \cdot \vec{w} &=& 0  \end{array} \right. \right) \) \( =  \vec{w} = (v_2u_3 - v_3u_2,\quad -[v_1u_3 - v_3u_1],\quad v_1u_2 - v_2u_1) \)

 

Isso quer dizer que, dado o sistema de equações em que \(\vec{v}\) compõe a primeira linha e \(\vec{u}\) a segunda (atente para esta ordem), o vetor \(\vec{w}\) é uma solução.

Note que, assim, fica nítido um padrão:

A coordenada 1 de \(\vec{w}\) só tem termos com índices 2 ou 3. O análogo vale para as demais.
Escrevemos sempre \(v_iu_j - v_ju_i\), com \(j > i\). Cabe lembrar o sinal de menos na segunda coordenada de \(\vec{w}\). Neste caso, \(\vec{v}\) deve ter dado origem à primeira linha do sistema de equações, porque, como veremos mais adiante, a ordem de \(\vec{v}\) e de \(\vec{u}\) impacta o resultado.

Queiramos ou não, isso define uma espécie de receita para produzirmos um vetor ortogonal a um certo par de vetores \(\vec{v}, \vec{u}\). Como cada coordenada envolve um produto de números, e como o resultado dessa "receita" é um vetor, decidiram defini-la como uma operação: O Produto Vetorial.

Simbolicamente,

\[
\vec{v} \times \vec{u} = (v_2u_3 - v_3u_2,\quad -[v_1u_3 - v_3u_1],\quad v_1u_2 - v_2u_1)
\], em que \(\vec{v}\) corresponde à primeira linha no sistema que resolvemos há pouco.

"Mas por que um \(\times\)?", você pode estar se perguntando...

Os matemáticos encontraram uma forma bem visual de calcular esse produto (que, em inglês, recebe o nome de Cross Product):



1. Escreva três linhas, cada qual com três colunas, entre duas barras verticais:

a. Na primeira, escreva \(\hat{i}\), \(\hat{j}\) e \(\hat{k}\), que representam, respectivamente, as coordenadas \(x\), \(y\) e \(z\).
b. Na segunda linha, para cada coluna, coloque a coordenada do primeiro vetor do produto (neste caso, \(\vec{v}\)) correspondente ao elemento na linha de cima. Por exemplo, embaixo de \(\hat{i}\) deve vir \(v_1\).
c. Na terceira, repita o processo para o segundo vetor do produto.

Por exemplo, para calcularmos \((1, 2, 3) \times (4, 5, 6)\), faremos

\[
\begin{vmatrix} \hat{i} & \hat{j} & \hat{k} \\ 1 & 2 & 3 \\ 4 & 5 & 6 \end{vmatrix}
\]

2. Para cada elemento da primeira linha, tape o restante da coluna e da linha em que este se encontra. Sobrarão quatro números.

No nosso caso,

\[
\hat{i} \to \begin{vmatrix} 2 & 3 \\ 5 & 6 \end{vmatrix}
\]


\[
\hat{j} \to \begin{vmatrix} 1 & 3 \\ 4 & 6 \end{vmatrix}
\]


\[
\hat{k} \to \begin{vmatrix} 1 & 2 \\ 4 & 5 \end{vmatrix}
\]

Cada um desses blocos produzirá a coordenada respectiva ao elemento da primeira linha associado — ou seja, ao taparmos a linha e a coluna de \(\hat{i}\), encontraremos o bloco respectivo à coordenada 1, e assim em diante. É preciso atentar, todavia, que, à semelhança do que tínhamos feito antes, será necessário colocar um sinal de menos na segunda coordenada:

\[ (1, 2, 3) \times (4, 5, 6) = 
\left(
\begin{vmatrix} 2 & 3 \\ 5 & 6 \end{vmatrix},\quad
-\begin{vmatrix} 1 & 3 \\ 4 & 6 \end{vmatrix},\quad
\begin{vmatrix} 1 & 2 \\ 4 & 5 \end{vmatrix}
\right)
\]

3. Para terminar, devemos fazer a "cruz" (\(\times\)): pegamos o termo no canto superior esquerdo e multiplicamos pelo inferior direito. Em seguida, calculamos a diferença entre esse valor e o produto dos outros dois números (trace mentalmente duas diagonais representando as multiplicações e você verá a "cruz"). Por exemplo,

\[
\begin{vmatrix} 2 & 3 \\ 5 & 6 \end{vmatrix} = 2 \cdot 6\;- 3 \cdot 5
\]

Assim,

\[ (1, 2, 3) \times (4, 5, 6) = 
\left(
\begin{vmatrix} 2 & 3 \\ 5 & 6 \end{vmatrix},\quad
-\begin{vmatrix} 1 & 3 \\ 4 & 6 \end{vmatrix},\quad
\begin{vmatrix} 1 & 2 \\ 4 & 5 \end{vmatrix}
\right)
\]

\(\Longleftrightarrow\) \[ (1, 2, 3) \times (4, 5, 6) = 
\left(
2 \cdot 6\;- 3 \cdot 5,\quad
-[1 \cdot 6\;- 3 \cdot 4],\quad
1 \cdot 5\; - 2 \cdot 4
\right)
\]

\(\Longleftrightarrow\) \[ (1, 2, 3) \times (4, 5, 6) = 
\left(
-3 ,
6,
-3
\right)
\]

Essa "cruz" explica o porquê do símbolo e do nome em inglês.

Para aqueles que conheçam matrizes, o processo que fizemos equivale a calcular o determinante \( \begin{vmatrix} \hat{i} & \hat{j} & \hat{k} \\ 1 & 2 & 3 \\ 4 & 5 & 6 \end{vmatrix} \) tomando certa liberdade poética em considerar \( \hat{i}, \hat{j},\hat{k} \) como se estes se comportassem à maneira de números reais. Seja como for, fato é que funciona.