Agora que definimos o produto vetorial, é hora de fazermos mais umas continhas, para deduzirmos algumas propriedades interessantes.

1. Antissimetria

\[\forall \vec{v}, \vec{u} \in \mathbb{R}^3, (\vec{v} \times \vec{u}) = - (\vec{u} \times \vec{v})
\]

Sim, é isso mesmo: nem todo produto é comutativo. Isso explica por que fui tão enfático com a ordem de \(\vec{v}\) e de \(\vec{u}\).

Para ver isso, notemos que

\[
\vec{v} \times \vec{u} = (v_2u_3 - v_3u_2,\quad -[v_1u_3 - v_3u_1],\quad v_1u_2 - v_2u_1)
\]

\(\Longleftrightarrow\) \[
\vec{v} \times \vec{u} = - (v_3u_2 - v_2u_3,\quad -[v_3u_1 - v_1u_3],\quad v_2u_1 - v_1u_2)
\] (colocando \(-1\) em evidência)

\(\Longleftrightarrow\) \[
\vec{v} \times \vec{u} = - (u_2v_3 - u_3v_2,\quad -[u_1v_3 - u_3v_1],\quad u_1v_2 - u_2v_1)
\] (reorganizando)

\(\Longleftrightarrow\) \[
\vec{v} \times \vec{u} = - (\vec{u} \times \vec{v})
\]

 

2. Homogeneidade

\[\forall \vec{v}, \vec{u} \in \mathbb{R}^3, \forall \lambda \in \mathbb{R}, (\lambda\vec{ v} \times \vec{u}) = (\vec{v} \times \lambda\vec{u}) = \lambda(\vec{v} \times \vec{u})
\]

Vamos provar isso:

\[
\left(\lambda\vec{v} \times \vec{u}\right) = ((\lambda v_2)u_3 - (\lambda v_3)u_2,\quad -[(\lambda v_1)u_3 - (\lambda v_3)u_1],\quad (\lambda v_1)u_2 - (\lambda v_2)u_1)
\]

\(\Longleftrightarrow\) \[
\left(\lambda\vec{v} \times \vec{u}\right) = (\lambda [v_2u_3 - v_3u_2],\quad -\lambda [v_1u_3 - v_3u_1],\quad \lambda [v_1u_2 - v_2u_1])
\]

\(\Longleftrightarrow\) \[
\left(\lambda\vec{v} \times \vec{u}\right) = \lambda (\vec{v} \times \vec{u})
\]

A igualdade \(\left(\vec{v} \times \lambda\vec{u}\right) = \lambda (\vec{v} \times \vec{u})\) se prova da mesma forma.

 

3. Aditividade

\[
\forall \vec{v}, \vec{u}, \vec{w} \in \mathbb{R}^3, \vec{v} \times (\vec{u} + \vec{w}) = (\vec{v} \times \vec{u}) + (\vec{v} \times \vec{w})
\]

Prova:

Digamos que \( \vec{v} = (v_1, v_2, v_3),\; \vec{u} = (u_1, u_2, u_3),\; \vec{w} = (w_1, w_2, w_3) \).

Logo,

\[
\vec{v} \times (\vec{u} + \vec{w}) = (v_2(u_3 + w_3) - v_3(u_2 + w_2),\quad -[v_1(u_3 + w_3) - v_3(u_1 + w_1)],\quad v_1(u_2 + w_2) - v_2(u_1 + w_1))
\]

 

Donde,

\[
\vec{v} \times (\vec{u} + \vec{w}) = (v_2u_3 - v_3u_2,\quad -[v_1u_3 - v_3u_1],\quad v_1u_2 - v_2u_1) + \\
\quad \quad \quad \quad \quad + \;(v_2w_3 - v_3w_2,\quad -[v_1w_3 - v_3w_1],\quad v_1w_2 - v_2w_1)
\]

\(\Longleftrightarrow\) \[
\vec{v} \times (\vec{u} + \vec{w}) = (\vec{v} \times \vec{u}) + (\vec{v} \times \vec{w})
\]

 

Note que isso vale para as duas posições:

\[
(\vec{u} + \vec{w}) \times \vec{v} =-[\vec{v} \times (\vec{u} + \vec{w})]
\]

\(\Longleftrightarrow\) \[
(\vec{u} + \vec{w}) \times \vec{v} =-[\vec{v} \times \vec{u} + \vec{v} \times \vec{w}]
\]

\(\Longleftrightarrow\) \[
(\vec{u} + \vec{w}) \times \vec{v} =-[-(\vec{u} \times \vec{v}) - (\vec{w} \times \vec{v})]
\]

\(\Longleftrightarrow\) \[
(\vec{u} + \vec{w}) \times \vec{v} =(\vec{u} \times \vec{v}) + (\vec{w} \times \vec{v})
\]

 

Conclusão

O produto vetorial se comporta de uma maneira bem parecida com o produto escalar, tirando o fato de que o vetorial não é comutativo.